\documentclass{subfiles}
\begin{document}

\begin{chapterimg}[height=\textheight]{icons/crucifixion}
    \Chapter{Horas Pascais}{}
\end{chapterimg}

\dotextit{%
Durante toda a Semana Luminosa (até o próximo Sábado após a Páscoa) estas Horas 
substituem as Orações Matinais e Noturnas, bem como o Ofício da Meia-
Noite, as Horas (I, III, VI, IX), e as Completas (Ofício Pós-ceia).}

\dotextit{%
Tendo acalmado todos os teus sentidos, com reverência começa tuas 
orações d’est’arte:}

Pelas orações de nossos santos Padres, ó Senhor Jesus Cristo, 
Deus nosso, tem piedade de nós. Amém.  

Cristo ressuscitou dos mortos, calcando a morte com a morte, e aos que estavam
nos sepulcros deu-lhes a vida. \thrice{}

\section*{Tom VI (pl. do II)}

Tendo contemplado a ressurreição de Cristo,/ adoremos \cross{} o santo Senhor Jesus,/ o
único sem pecado./ A Tua cruz adoramos \cross{}, ó Cristo, / e a Tua santa ressurreição
cantamos e glorificamos,/ pois que Tu és o nosso Deus./ Outro não conhecemos
senão a Ti,/ o Teu nome clamamos./ Vinde todos os fiéis,/ adoremos \cross{} a santa
ressurreição de Cristo,/ pois pela cruz veio a alegria a todo o mundo,/ sempre
bendizendo o Senhor,/ cantemos a Sua ressurreição:/ pois sofreu a crucificação/
e aniquilou a morte com a morte. \thrice{}

\section*{Hypakoí, Tom IV (tradicionalmente cantado no VIII)}

Tendo antecedido a aurora junto com Maria,/ e encontrando removida a pedra do
sepulcro, elas escutaram do anjo:/ «Ele habita na luz sempiterna./ Porquanto,
por que entre os mortos procuram-No como se fosse humano?/ Vede a mortalha
fúnebre, ide rapidamente/ e ao mundo proclamai que ressuscitou o Senhor, matando
a própria morte,/ pois que é o Filho de Deus, Quem salva o gênero humano». 

\section*{Kondáquio, Tom VIII (pl. do IV)}

Mesmo tendo ao sepulcro descido, ó Imortal,/ Tu 
aniquilaste o poder do inferno,/ e como triunfador 
ressuscitaste, ó Cristo Deus,/ bradando às mulheres 
miróforas: Rejubilai-vos!,/ e aos Teus apóstolos dando-lhes 
a paz,/ e aos decaídos concedendo a ressurreição.  

\dotextit{E os seguintes tropários:}

Com o corpo no sepulcro e no inferno com a alma, assaz como Deus,/ e no paraíso
com o ladrão e no trono estiveste com o Pai e o Espirito, ó Cristo,/
plenificando tudo, ó Inefável. 

\doxology{}

Como vivaz e mais belo que o paraíso,/ e verdadeiramente 
mais resplandecente do que qualquer palácio real revelou-se 
o Teu sepulcro, ó Cristo,/ a fonte da nossa ressurreição. 

\nowandever{}

Ó santificada e divina Morada do Altíssimo, rejubila-te!/ 
Visto que por ti, ó Deípara, o júbilo é dado àqueles que te 
bradam:/ Bendita és tu entre as mulheres, Senhora 
imaculadíssima. 

\mercy{} \repeatn{40}

Mais honorável que os querubins e incomparavelmente mais 
gloriosa que os serafins,/ que imaculadamente deste à luz o 
Verbo de Deus,/ ó Deípara, nós te magnificamos! 

\Doxology{}

\mercy{} \thrice{}

Senhor, abençoa. 

Senhor Jesus Cristo, Filho de Deus, que ressuscitaste dos 
mortos, pelas orações da Tua Puríssima Mãe, dos nossos 
veneráveis e teóforos Padres e de todos os santos, tem 
piedade de nós e salva-nos, pois que és Bom e Filantropo. 
Amém. 

Cristo ressuscitou dos mortos, calcando a morte com a 
morte, e aos que estavam nos sepulcros deu-lhes a vida. \thrice{}

E concedeu-nos a vida eterna. Adoramos a Sua ressurreição 
ao terceiro dia. 

\end{document}
