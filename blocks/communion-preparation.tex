\documentclass{subfiles}

\begin{document}

\chapter{Orações de Preparaçao para Comunhão}

Pelas orações dos nossos santos pais, Ó Cristo Nosso Deus, tem
piedade de nós. Amém.

Glória a Ti, Ó nosso Deus, glória a Ti!

\comforter{}

\trisagion{} \thrice{}

\Doxology{}

Santíssima Trindade tem piedade de nós. Senhor, purifica-nos dos
nossos pecados. Soberano, perdoa-nos as nossas iniquidades. Tu que és Santo,
cura, pelo Teu Nome, as nossas enfermidades e visita-nos.

\mercy{} \thrice{}

\Doxology{}

\ourFather{}

\mercy{} \thrice{}

\letusworship{}

\section*{Salmo 22}

O Senhor é o meu pastor; nada me faltará. Deitar-me faz em verdes pastos,
guia-me mansamente a águas tranquilas. Refrigera a minha alma; guiame pelas
veredas da justiça por amor do seu nome. Ainda que eu andasse pelo vale da
sombra da morte, não temeria mal algum, porque tu estás comigo; a tua vara e o
teu cajado me consolam. Preparas uma mesa perante mim na presença dos meus
inimigos, unges a minha cabeça com óleo, o meu cálice transborda. Certamente que
a bondade e a misericórdia me seguirão todos os dias da minha vida; e habitarei
na Casa do Senhor por longos dias.

\section*{Salmo 23}

Do Senhor é a terra e a sua plenitude, o mundo e aqueles que nele habitam.
Porque ele a fundou sobre os mares e a firmou sobre os rios. Quem subirá ao
monte do Senhor ou quem estará no seu lugar santo? Aquele que é limpo de mãos e
puro de coração, que não entrega a sua alma à vaidade, nem jura enganosamente.
Este receberá a bênção do Senhor e a justiça do Deus da sua salvação. Esta é a
geração daqueles que buscam, daqueles que buscam a tua face, Ó Deus de Jacó.
Levantai, ó portas, as vossas cabeças; levantai-vos, ó entradas eternas, e
entrará o Rei da Glória. Quem é este Rei da Glória? O Senhor Poderoso e
poderoso, o Senhor poderoso na guerra. Levantai, ó portas, as vossas cabeças;
levantai-vos, ó entradas eternas, e entrará o Rei da Glória. Quem é este Rei da
Glória? O Senhor dos Exércitos; ele é o Rei da Glória.

\section*{Salmo 115}

Cri; por isso, falei: estive muito aflito. Eu dizia na minha precipitação: todo
homem é mentira. Que darei eu ao Senhor por todos os benefícios que me tem
feito? Tomarei o cálice da salvação e invocarei o nome do Senhor. Pagarei os
meus votos ao Senhor, agora, na presença de todo o seu povo. Preciosa é à vista
do Senhor a morte dos seus santos. Ó Senhor, deveras sou teu servo; sou teu
servo, filho da tua serva; soltaste as minhas ataduras. Oferecer-te-ei
sacrifícios de louvor e invocarei o nome do Senhor. Pagarei os meus votos ao
Senhor; que eu possa fazê-lo na presença de todo o meu povo, nos átrios da Casa
do Senhor, no meio de ti, ó Jerusalém!

% TODO: maybe include a text break in here.

\Doxology{}

Aleluia, aleluia, aleluia. Glória a Ti, Ó Deus. \dotextit{(3 vezes)}

Senhor, tem piedade. \dotextit{(3 vezes)}

\section*{Tropário, Tom 8}

Senhor, não consideres as minhas transgressões, Tu que nasceste de uma Virgem, e
purifica o meu coração. Faz dele um templo para o Teu Corpo e Sangue imaculados.
Não me lances fora da Tua presença, Tu cuja misericórdia não tem medida.

\doxology{}

Como posso, indigno que sou, ousar aproximar-me para a Comunhão
de Teus Santos Dons? Pois que se ousasse aproximar-me de Ti junto daqueles
que são dignos, meus trajes me trairiam, pois não são trajes de festa, causando
a condenação da minha alma pecadora. Lava, Senhor, a sujeira da minha alma,
e salva-me pois Tu amas a humanidade.

\nowandever{}

Infinitamente multiplicados são os meus pecados, ó Mãe de Deus; a ti dirijo-me,
ó Puríssima, implorando salvação. Visita minha alma enfraquecida, e roga a teu
Filho e nosso Deus que me conceda o perdão pelo mal que fiz, ó tu, única
abençoada.

\dotextit{Durante a Santa e Grande Quaresma dizer:}

Enquanto os gloriosos apóstolos eram iluminados na lavagem dos pés,
Judas, o ímpio, era atingido e obscurecido pelo amor ao dinheiro. E aos juízes
sem lei entregou a Ti, o Justo juiz. Vê, ó amante do dinheiro, aquele que por
este amor enforcou-se; foge desta alma insaciável que ousou tais coisas contra
o Soberano. Ó Tu que és bom para com todos, Senhor, glória a Ti.

\section*{Salmo 50}

Tem misericórdia de mim, Ó Deus, segundo a tua benignidade; apaga
as minhas transgressões, segundo a multidão das tuas misericórdias. Lava-me
completamente da minha iniquidade e purifica-me do meu pecado. Porque eu
conheço as minhas transgressões, e o meu pecado está sempre diante de mim.
Contra ti, contra ti somente pequei, e fiz o que a teus olhos é mal, para que
sejas justificado quando falares e puro quando julgares. Eis que em iniquidade
fui formado, e em pecado me concebeu minha mãe. Eis que amas a verdade
no íntimo, e no oculto me fazes conhecer a sabedoria. Purifica-me com
hissopo, e ficarei puro; lava-me, e ficarei mais alvo do que a neve. Faz-me ouvir
júbilo e alegria, para que gozem os ossos que tu quebraste. Esconde a tua face
dos meus pecados e apaga todas as minhas iniquidades. Cria em mim, Ó Deus,
um coração puro e renova em mim um espírito reto. Não me lances fora da
tua presença e não retires de mim o teu Espírito Santo. Torna a dar-me a
alegria da tua salvação e sustém-me com um espírito voluntário. Então,
ensinarei aos transgressores os teus caminhos, e os pecadores a Ti se
converterão. Livra-me dos crimes de sangue, Ó Deus, Deus da minha salvação,
e a minha língua louvará altamente a tua justiça. Abre, Senhor, os meus lábios,
e a minha boca entoará o teu louvor. Porque te não comprazes em sacrifícios,
senão eu os daria; tu não te deleitas em holocaustos. Os sacrifícios para Deus
são o espírito quebrantado; a um coração quebrantado e contrito não
desprezarás, Ó Deus. Abençoa a Sião, segundo a tua boa vontade; edifica os
muros de Jerusalém. Então, te agradarás de sacrifícios de justiça, dos
holocaustos e das ofertas queimadas; então, se oferecerão novilhos sobre o
teu altar.

\dotextit{E imediatamente após:}

\section*{O Cânone para a Santa Comunhão. Segundo Tom.}

\section*{ODE 1}

\eirmos{}Vinde, ó povo, cantemos um hino a Cristo nosso Deus, que dividiu o mar
e guiou Seu povo para fora da servidão do Egito, pois Ele é glorificado.
76

\refrain{}Cria em mim, Ó Deus, um coração puro e renova em mim um espírito reto.

Possa o Teu Santo Corpo ser para mim o Pão da vida eterna, Ó Senhor compassivo,
e o Teu precioso Sangue ser a cura para toda a enfermidade. Refrão: Não me
lances fora da Tua presença nem retires de mim o Teu Espírito Santo.

Corrompido por meus feitos indecentes, eu, o miserável, sou indigno da comunhão
do Teu puríssimo Corpo e divino Sangue, Ó Cristo, que Tu Te dignaste
conceder-me.

\Doxology{}

Ó abençoada noiva de Deus, ó solo fértil de onde brotou o Milho não semeado para
a salvação do mundo, concede-me que seja salvo ao comungar d’Ele.

\section*{ODE 3}

\eirmos{}Estabelecendo-me na rocha da fé, Tu abriste a minha boca sobre os meus
inimigos, por isto meu espírito rejubila quando canto: Não há santo como o nosso
Deus, nem justo como Tu.

Ó Senhor, cria em mim, Ó Deus, um coração puro e renova em mim um espírito reto.

Concede-me lágrimas, Ó Cristo, para a purificação de meu coração impuro, para
que limpo e com a consciência tranquila, eu possa aproximar-me com fé e temor, Ó
Soberano, da Comunhão dos Teus divinos Dons.

Não me lances fora da Tua presença nem retires de mim o Teu Espírito Santo.

Que o Teu puríssimo Corpo e divino Sangue sejam para a remissão dos meus
pecados, para a comunhão com o Espírito Santo e para a vida eterna, Ó Amigo dos
homens, e para o afastamento das paixões e sofrimentos.

\Doxology{}

Ó Tu, santíssimo tabernáculo do Pão da Vida que por misericórdia vieste do alto
para dar a vida nova ao mundo, concede até mesmo a mim, indigno que sou, que
como d’Ele com temor, e viva.

\section*{ODE 4}

\eirmos{}De uma Virgem vieste, não como embaixador ou como um anjo, mas o próprio
verdadeiro Senhor encarnado, e salvaste a mim, inteiramente homem. Por isto a Ti
clamo: Glória ao Teu poder, Senhor.

Cria em mim, Ó Deus, um coração puro e renova em mim um espírito reto.

Ó Tu que encarnaste para nossa salvação, Misericordioso, Tu quiseste ser
sacrificado como um cordeiro pelos pecados da humanidade. Por isto rogo a Ti que
apagues também os meus pecados.

Não me lances fora da Tua presença nem retires de mim o Teu Espírito Santo.

Cura as feridas da minha alma, Ó Senhor, e santifica-me por inteiro, e concede a
mim, miserável que sou, participar de Tua divina Mística Ceia.

\Doxology{}

Propicia-me também aquele que veio de teu ventre, ó Senhora, e conserva
imaculado e sem culpa, a mim teu servo, para que obtendo a Pérola espiritual eu
possa ser santificado.

\section*{ODE 5}

\eirmos{}Ó Senhor, Doador da luz e Criador das eras, guia-nos na luz dos Teus
mandamentos, pois não conhecemos outro Deus.

Cria em mim, Ó Deus, um coração puro e renova em mim um espírito reto.

Como predisseste, Ó Cristo, que assim seja para comigo, Teu servo
iníquo, e como prometeste habita em mim; para isto, vê, como do Teu divino
Corpo e bebo do Teu Sangue.

Não me lances fora da Tua presença nem retires de mim o Teu Espírito Santo.

Ó Verbo de Deus e Deus, possa a brasa viva do Teu Corpo ser luz para
mim que estou nas trevas, e o Teu Sangue lave a minha alma impura.

\Doxology{}

Ó Maria, mãe de Deus, precioso tabernáculo de perfume, pelas tuas
orações, torna-me um vaso escolhido, para que eu possa partilhar do
Sacramento de teu Filho.

\section*{ODE 6}

\eirmos{}Arrastado para o abismo do pecado, apelo para o insondável abismo da
Tua compaixão: Levanta-me da corrupção Ó Deus.

Cria em mim, Ó Deus, um coração puro e renova em mim um espírito reto.

Ó Salvador, santifica o meu pensamento, minha alma, meu coração e meu corpo, e
concede-me que sem condenação, Soberano, possa aproximarme dos Teus temíveis
Mistérios.

Não me lances fora da Tua presença nem retires de mim o Teu Espírito Santo.

Concede-me que eu possa livra-me das paixões e ser assistido pela Tua Graça,
e ser fortalecido pela comunhão de Teus Santos Mistérios, Ó Cristo.

\Doxology{}

Ó Santo verbo de Deus e Deus, santifica-me por inteiro pois agora me apresento
perante os Teus divinos Mistérios, pelas orações da Tua Santíssima Mãe.

Senhor, tem piedade. \dotextit{(3 vezes)}

\Doxology{}

\section*{Kontákion, Tom 2:}

Não me consideres indigno, Ó Cristo, de receber agora o Pão que é o Teu Corpo e
o Teu divino Sangue, e de compartilhar, Ó Soberano, dos Teus puríssimos e
temíveis Mistérios, apesar de ser o miserável que sou. Não permitas que isto
seja para mim motivo de julgamento, mas para a vida imortal e eterna.

\section*{ODE 7}

\eirmos{}As sábias crianças não se prostraram perante o ídolo de ouro, mas
foram lançadas às chamas e insultaram os deuses pagãos. Em meio às chamas
clamaram e o anjo fez chover o orvalho sobre eles: As preces de vossos lábios
já foram ouvidas.

Cria em mim, Ó Deus, um coração puro e renova em mim um espírito reto.

Que a comunhão dos Teus Mistérios imortais, a fonte de bênçãos, Ó
Cristo, possa ser para mim agora luz, e vida, e temperança, e que eu possa
prosperar e crescer nas diviníssimas virtudes, Ó Único Bom, para que eu possa
glorificar-Te.

Não me lances fora da Tua presença nem retires de mim o Teu Espírito Santo.

Para que possa ser libertado das paixões, dos inimigos, necessidades,
e de todo o sofrimento, aproximo-me agora com temor, amor e respeito, dos
Teus imortais e divinos Mistérios, Ó Amigo do homem. Concede-me Te louvar:
Bendito és Tu, Ó Senhor Deus de nossos pais.

\Doxology{}

Ó Tu que és cheia de graça, que incompreensivelmente deste à luz a Cristo
Salvador, eu, teu servo impuro, rogo-te, ó Toda Pura: Lava-me de toda a impureza
da carne e do espírito, a mim que agora me aproximo dos puríssimos Mistérios.

\section*{ODE 8}

\eirmos{}O Deus que desceu sobre os jovens hebreus na fornalha e transformou
as chamas em orvalho, louvai-O como Senhor, ó obras Suas, e exaltai-O por
todos os séculos.

Cria em mim, Ó Deus, um coração puro e renova em mim um espírito reto.

De Teus celestes e temíveis Mistérios, Ó Cristo, de Tua divina Mística
Ceia, concede-me partilhar, Deus meu Salvador, até mesmo a mim, miserável
que sou.

Não me lances fora da Tua presença nem retires de mim o Teu Espírito Santo.

Buscando refúgio em Teu amor, Ó Único Bom, com temor clamo a Ti: Habita em mim,
Ó Salvador, e eu, como disseste, em Ti. Para isto, na esperança da Tua
misericórdia, como de Teu Corpo e bebo de Teu Sangue.

\Doxology{}

Tremo consumindo-me em fogo, deixa-me ser consumido como cera e grama. Ó temível
Mistério! Ó amor de Deus! Como pode ser que eu, que não sou senão argila, ao
comungar o divino Corpo e Sangue, seja feito incorruptível?

\section*{ODE 9}

\eirmos{}O Filho do Pai incriado, Deus e Senhor, apareceu para nós encarnado
da Virgem, para iluminar aqueles que estavam nas trevas, e para reunir os
dispersos. Por isso, nós te glorificamos, ó Mãe de Deus.

Cria em mim, Ó Deus, um coração puro e renova em mim um espírito reto.

Provai e comei, Isto é Cristo! Para nossa salvação, o senhor fez- se
como nós, e ofereceu-se em sacrifício a Seu Pai, e para sempre será sacrificado,
santificando, assim, aqueles que d’Ele participam.

Não me lances fora da Tua presença nem retires de mim o Teu Espírito Santo.

Que eu seja santificado em alma e corpo, Ó Soberano, que eu seja
iluminado, salvo, e torne-me a Tua morada pela comunhão dos Teus santos
Mistérios, tendo a Ti com o Pai e o Espírito habitando em mim, Ó Benfeitor
cheio de misericórdia.

\doxology{}

Possam o Teu Corpo e o Teu preciosíssimo Sangue, meu Salvador, ser em mim como
fogo e luz, consumindo a substância do pecado, e queimando os espinhos das
paixões, iluminando-me por inteiro para que eu possa louvar a Tua Divindade.

\nowandever{}

Deus tomou carne do teu puro sangue; por isto todas as gerações te
louvam, ó Senhora, e multidões de seres celestes te glorificam, pois através de
ti, revestido da nossa natureza humana, foi claramente visto Aquele que regula
todas as coisas.

\dotextit{E logo em seguida:}

Verdadeiramente é digno que Te bendigamos, ó Bem-aventurada Mãe
do nosso Deus. Mais venerável que os querubins e incomparavelmente mais
gloriosa que os Serafins, deste a luz o verbo de Deus, conservando intacta a a
tua virgindade. Nós te glorificamos, ó Mãe de Deus.

\trisagion{} \thrice{}

Santíssima Trindade, tem piedade de nós. Senhor, purifica-nos dos
nossos pecados. Soberano, perdoa-nos as nossas iniquidades. Tu que és Santo,
cura, pelo Teu Nome, as nossas enfermidades e visita-nos.

\mercy{} \thrice{}

\Doxology{}

\ourFather{}

\dotextit{Tropário do Dia, se for a festa da Natividade de Cristo. Se for
Domingo, o tropário daquele Domingo no respectivo tom. Se não, serão estes a
seguir:}

Tom 6: Tem piedade de nós, Senhor, tem piedade, pois sem nenhuma outra
defesa, nós, pecadores, oferecemos esta oração a Ti como Soberano: tem
piedade de nós.

\doxology{}

Senhor, tem piedade de nós, pois em Ti depositamos a nossa esperança, não Te
enfureças conosco, nem Te lembres das nossas iniquidades, mas olha para nós
agora pois Tu és compassivo, e livra-nos dos nossos inimigos pois Tu és o nosso
Deus e nós somos o Teu povo; somos obra das Tuas mãos e clamamos o Teu nome.

\nowandever{}

Abre para nós as portas da compaixão, bendita Mãe de Deus, não nos
deixes perecer pois em ti depositamos nossa esperança; que pela tua
intercessão possamos ser libertados das adversidades pois tu és a salvação do
povo Cristão.

\mercy{} \repeatn{40}

\dotextit{(Fazer tantas reverências — pequenas e grandes metanóias — quanto se queira)}

\dotextit{E em seguida:}

Se desejas, ó homem, comungar o Corpo do Soberano,
Aproxima-te com temor, deixa-te arder, pois Ele é chama.
E quando fores tomar do Divino Sangue na comunhão,
Reconcilia-te, primeiro, com aqueles que te causaram o mal
Só então ousa comungar o Místico Alimento.

\dotextit{E ainda:}

Antes de participar do maravilhoso Sacrifício do Corpo doador da vida do
Soberano, depois disto ora com tremor.

\section*{Oração de São Basílio, O Grande, 1}

Senhor Jesus Cristo, nosso Deus Soberano, fonte da vida e imortalidade, Criador
de todos os seres visíveis e invisíveis, Filho eterno e unigênito de Deus
eterno, que pela Tua inefável benevolência desejaste nos últimos dias Te
revestir de corpo, ser crucificado e sepultado por causa de nós, indignos e
maldosos, e renovar com Teu Sangue, nossa natureza corroída pelo pecado: Tu, Rei
imortal, receba a penitência deste pecador, reclina-Te para mim e ouve-me:
pequei, Senhor, pequei perante o Céu e perante Ti, e já não sou digno de olhar
para as alturas de Tua glória. Enfureci Tua bondade, transgredi as Tuas leis e
desobedeci Tua vontade. Mas Tu, Bondoso Senhor, paciente e misericordiosíssimo,
não me deixastes perecer com toda a minha iniquidade, sempre aguardando o meu
arrependimento. Tu disseste, Ó Senhor, a Teus profetas: Não desejo a morte do
pecador, mas que ele se arrependa e viva eternamente. Tu não desejas, Ó
Soberano, que a obra de Tuas mãos pereça, nem Te apetece a perdição do homem,
mas desejas que todos se salvem e cheguem ao conhecimento da verdade. Por isso e
eu, mesmo sendo indigno do Céu, da Terra e desta vida passageira, pois estou
submerso no pecado, afundado nas paixões e maculei a Tua imagem. Porém sou obra
e criatura de Tuas mãos e por isso não me desespero da salvação, mas me entrego
todo à Tua infinita misericórdia. Receba-me, Ó Senhor benevolente, como
recebeste a meretriz, o ladrão, o publicano e o pecador, e perdoa todos os meus
pesados pecados, Tu que redimiste os pecados do mundo, que curas as doenças dos
sofredores; que chamas a Si os trabalhadores e sofredores e os consolas; que
vieste para chamar não os justos, mas os pecadores para o arrependimento e
penitência. Limpa-me de toda a iniquidade do corpo e da alma, e ensina-me a
praticar a santidade do Teu temor. E, tendo assim, limpa a consciência, e
recebendo os Teus Santos Dons, me junte aos Teus Santíssimos Corpo e Sangue e
tenha a Ti vivendo em mim com o Pai e o Teu Espírito Santo. Ó Senhor Jesus
Cristo, Deus meu, que esta comunhão dos Teus Santos Dons não seja para meu
juízo, nem para enfraquecimento do corpo e da minha alma, pois sei que estou
recebendo-Os indignamente. Porém, Senhor, até o meu último suspiro permita-me
receber sem condenação os Teus Santos Mistérios, unir-me ao Espírito Santo,
receber a vida eterna e uma sentença favorável no Teu Juízo Final. Para que,
assim, eu esteja entre os Teus escolhidos e seja participante dos bens eternos,
que preparaste, Senhor, para que os que Te amam, nos quais És bendito por todos
os séculos. Amém.

\section*{Oração de nosso Pai entre os Santos, João Crisóstomo, 2:}

Senhor meu Deus, sei que não sou digno de que entres sob o teto do templo da
minha alma, pois tudo está vazio e decaído e não tens em mim um lugar digno para
repousares a Tua cabeça. Mas assim como do alto Tu desceste para nossa salvação,
desce também agora até a minha baixeza, e assim como consentiste deitar-Te numa
gruta e na manjedoura de animais mudos, consente também deitar-Te na manjedoura
da minha alma irracional e entrar em meu corpo impuro. Assim como não Te
recusaste a entrar e ceiar com os pecadores na casa de Simão, o leproso,
concede-me entrares na casa da minha alma leprosa e pecadora. Assim como não
rejeitaste a meretriz e pecadora como eu quando ela aproximou-se e tocou-Te, sê
compassivo também para comigo quando eu me aproximar e tocar-Te. Assim como não
Te enfureceste contra os lábios dela, impuros e indignos, que Te beijaram,
também não Te enfureças contra os meus lábios indignos, contra a minha boca
abominável e impura, nem contra minha língua poluída e suja. Permite que a brasa
ardente do Teu santíssimo Corpo e do Teu precioso Sangue seja para minha
santificação e iluminação, saúde para a minha alma e meu corpo, para o alívio da
carga de tantos pecados, para a preservação contra as artimanhas do demônio,
para a expulsão e a proibição dos meus hábitos vis , para mortificação das
paixões, para a manutenção dos Teus mandamentos, para a aplicação da Tua divina
graça, para a aquisição do Teu reino. Pois não é com escárnio que me aproximo de
Ti, Ó Cristo Deus, mas como alguém que crê na Tua inefável bondade, e para que
não me torne presa do lobo espiritual por abster-me da Tua comunhão. Por isto Te
rogo, Ó Único Santo: Soberano, santifica a minha alma e o meu corpo, minha mente
e o meu coração, meu ventre e meu interior e renova-me inteiramente. Instala em
meus membros o temor de Ti, tornando inalienável para mim a Tua santificação. Sê
o meu socorro e defesa, guiando a minha vida em paz. Concede-me também estar a
Tua direita com os Teus santos, pelas súplicas e intercessões de Tua puríssima
Mãe, dos Teus ministros imateriais e hostes imaculadas, e de todos os santos que
pelos séculos têm sido agradáveis a Ti. Amém.

\section*{Oração de Simeão Metafrastes, 3}

Senhor puro e sem pecado que pela inefável compaixão do Teu amor
pelos homens tornaste toda nossa substância do puro e virgem sangue
daquela que sobrenaturalmente carregou-Te pela descida do Espírito Divino e
boa vontade do Pai eterno; Ó Cristo, Sabedoria de Deus, Paz e Poder, Tu que
ao assumires a nossa natureza tomaste sobre Ti Tua vivificante e salvadora
Paixão — a Cruz, os cravos, a lança e a morte: mortifica as paixões do meu
corpo que corrompem a minha alma. Tu que pelo Teu sepultamento tornaste
cativo o reino do hades, sepulta com bons pensamentos meus esquemas
malignos e destrói os espíritos do mal. Tu, que pela Tua vivificante
Ressurreição ao terceiro dia ergueste nossos ancestrais decaídos, ergue a mim
que escorrego em direção ao pecado, colocando a minha frente os caminhos
do arrependimento. Tu que pela Tua gloriosíssima Ascensão deificaste a carne
que tomaras e a honraste com um lugar à direita do Pai, concede-me que, ao
participar dos Teus santos Mistérios, eu obtenha também um lugar à Tua
direita junto aqueles que foram salvos. Tu que pela descida do Teu Espírito, o
Consolador, tornaste Teus santos discípulos vasos dignos, mostra-me também
como ser um receptáculo para Sua descida. Tu que virás de novo para julgar o
mundo com justiça, digna-Te a permitir que eu também Te encontre, nos céus,
meu Juiz e meu Criador, com todos os Teus santos; que eu possa
incessantemente glorificar e louvar a Ti, e Teu Santíssimo, bom e vivificante
Espírito, agora e sempre e pelos séculos dos séculos. Amém.

\section*{Do Divino Damasceno, 4}

Ó Soberano, Senhor Jesus Cristo, nosso Deus, Tu que és o Único com autoridade
para perdoar os pecados dos homens: Tu, que és Bom e amas a humanidade, releva
todas as minhas ofensas, quer tenham sido cometidas com conhecimento ou em
ignorância. E concede-me participar sem perigo de condenação dos Teus Divinos,
gloriosos, imaculados e vivificantes Mistérios: não como opressão, nem castigo,
nem para aumento dos pecados, mas para purificação e santificação, e como
promessa de vida e do reino que há de vir, como baluarte e socorro, para a
destruição dos inimigos, e para apagar as minhas muitas transgressões. Pois Tu
és um Deus de misericórdia, compaixão e amor pela humanidade, a Ti damos glória,
com o Pai e o Espírito Santo, agora e sempre e pelos séculos dos séculos. Amém.

\section*{De São Basílio, o Grande, 5}

Senhor, sei que indignamente participo do Teu imaculado Corpo e do Teu precioso
Sangue, e que sou culpado, e assim como e bebo condenação para mim não
reconhecendo Teu Corpo e Teu Sangue, nem Cristo e meu Deus; mas tirando coragem
da Tua compaixão aproximo-me de Ti que disseste: quem come a Minha Carne e bebe
Meu Sangue, estará em Mim e Eu nele. Mostra a Tua compaixão, Senhor, e não me
acuses, pecador que sou, mas trata-me segundo a Tua misericórdia; e que estes
Santos Dons sejam para a cura, a purificação, a iluminação, a preservação, a
salvação e a santificação da alma e do corpo; para o afastamento de toda a
fantasia, das práticas maldosas, e das atividades que o demônio mentalmente
articula sobre meus membros; para a confiança e amor por Ti, para a correção da
vida, para a firmeza, o crescimento das virtudes e da perfeição, para o
cumprimento dos mandamentos, para a comunhão do Espírito Santo, como condição
para a vida eterna, como defesa aceitável no Teu temível tribunal e não para
julgamento ou condenação.


\section*{Uma oração de São Simeão, o novo Teólogo, 6}

De lábios imundos, de um coração abominável, de uma língua impura, de uma alma
obscurecida, aceita, Ó Cristo, a minha súplica, e não me desprezes, nem as
minhas palavras, nem os meus caminhos, nem a minha falta de vergonha.
Concede-me, meu Cristo, humildemente dizer o que desejo. Ou antes, ensina-me o
que devo fazer e dizer. Pequei mais do que a mulher pecadora que, sabendo onde
Te encontravas, Trouxe-Te mirra e ousou ungir Teus pés, meu Deus, meu Soberano e
meu Cristo. Ó Verbo, assim como não a rejeitaste quando ela aproximou-se de Ti,
não Te enfureças comigo, mas concede-me Teus pés para que Eu Os seque e beije, e
que minhas lágrimas, como preciosa mirra, ousem ungi-Los. Lava-me e purifica-me
com as minhas lágrimas, Ó Verbo, apaga as minhas transgressões e concede-me o
perdão. Tu conheces a multidão das minhas maldades, Tu também conheces as minhas
dores e curas as minhas feridas; conheces também a minha fé e observas as minhas
boas intenções e ouves o meu pranto. Nada Te é oculto, nem as lágrimas, nem uma
parte dela. Minhas ações ainda não cometidas Tu já as viste, e em Teu livro
mesmo aquilo que ainda não foi feito está escrito por Ti. Vê a minha baixeza, vê
a minha labuta, como é grande, e retira de mim todos os meus pecados, Ó Deus de
tudo: com um coração puro, tremor em meus pensamentos e uma alma contrita, possa
eu participar dos Teus imaculados e santíssimos Mistérios, pelos quais, aquele
que come e bebe com pureza de coração é prontamente deificado. Tu disseste,
Soberano: quem come a Minha Carne e bebe Meu Sangue está em Mim e Eu nele.
Verdadeiras são todas as palavras de meu Soberano e Deus; pois qualquer que
participe da divina e deificante graça não está mais sozinho mas está Contigo,
meu Cristo, a Luz tríplice que ilumina o mundo. E possa eu nunca ficar sozinho
sem Ti, Doador da vida, meu Fôlego, minha Vida, meu Júbilo, Salvação do mundo,
por isto lanço-me em Tua direção, como podes ver, com lágrimas e a alma
contrita. Suplico-Te, Ó Resgate de minhas ofensas, que me recebas e que eu possa
participar de Teus vivificantes e perfeitos Mistérios sem risco de condenação,
que Tu permaneças comigo, triplamente indigno que sou, como Tu prometeste, para
que o enganador, encontrando-me sem a Tua graça, ardilosamente não se apodere de
mim, e tendo me logrado, lance-me para longe das Tuas deificantes palavras.
Assim prostro-me diante de Ti e com fervor Te clamo: como recebeste o filho
pródigo e a mulher pecadora, Ó Compassivo, recebe também a mim, pródigo e
devasso que sou. Com a alma contrita venho a Ti. Sei, meu Salvador, que ninguém
pecou contra Ti como eu, nem cometeu as iniquidades que cometi. Mas sei também
que nem a magnitude das minhas ofensas, nem a multidão dos meus pecados
ultrapassam o abundante sofrimento do meu Deus e Seu ilimitado amor pela
humanidade; mas misericordiosamente Tu purificas e iluminas aqueles que se
arrependem, e os fazes partícipes da luz da Tua divindade sem restrição. E
estranho aos anjos e a mente dos homens tu conversaste com eles muitas vezes
como com Teus verdadeiros amigos. Estas coisas tornam-me arrojado, elas dão-me
asas, Ó Cristo. E tirando coragem da riqueza das Tuas bênçãos sobre nós,
rejubilando e tremendo, participo do Fogo, eu que sou erva. Estranha maravilha!
Ardo sem ser consumido, como a sarça de outrora. Com o pensamento e o coração
agradecidos, com gratidão em meus membros, alma e corpo, louvo-Te agora, e
glorifico-Te, meu Deus, pois bendito és Tu, agora e pelos séculos.

\section*{Outra oração de S. J. Crisóstomo, 7}

Ó Deus, desata, apaga e perdoa as minhas transgressões pois pequei
contra Ti, por palavras, atos e pensamento, voluntária ou involuntariamente,
consciente ou inconscientemente; perdoa-me tudo pois Tu és bom e amas a
humanidade. E pela intercessão da Tua puríssima Mãe, de Teus ministros
noéticos, das santas hostes e de todos os santos que através dos séculos Te
foram agradáveis, concede-me sem perigo de condenação Teu santo e
imaculado Corpo e precioso Sangue, para a cura da alma e do corpo e para a
purificação dos meus maus pensamentos. Pois a Ti pertence o reino, o poder
e a glória, com o Pai e o Espírito Santo, agora e sempre e pelos séculos dos
séculos. Amém.

\section*{Do mesmo santo, 8}

Soberano e Senhor, não sou digno que entres sob o teto da minha alma, mas porque
amas a humanidade, ouso aproximar-me de Ti: vem e habita em mim. Tu ordenaste:
devo abrir as portas que Tu criaste de maneira a que Tu entres e ilumines o meu
pensamento obscurecido. Creio ser este o Teu desejo pois tu não rejeitaste a
mulher pecadora quando a Ti ela veio em lágrimas, nem rejeitaste o publicano
arrependido, nem o ladrão que reconheceu Teu Reino, nem deixaste o acusador
arrependido entregue a si mesmo; todos eles vieram ter Contigo arrependidos e Tu
os contaste dentre os Teus amigos, Ó Tu que és o Único bendito, agora e sempre,
e eternamente. Amém.

\section*{Do mesmo Santo, 9}

Senhor Jesus Cristo meu Deus, desata, apaga, lava e perdoa o Teu servo pecador,
inútil e indigno, as minhas transgressões e ofensas e quedas no pecado que
cometi contra Ti desde a minha juventude até este dia e hora, consciente ou
inconscientemente, em palavras e atos, em pensamento e imaginação, por hábito e
com todos os meus sentidos. E pela intercessão daquela que sem semente deu-Te à
luz, a puríssima e Sempre Virgem Maria, Tua Mãe, minha única esperança, mediação
e salvação, concede-me sem perigo de condenação participar dos Teus imaculados,
imortais, vivificantes e maravilhosos Mistérios, para a remissão dos pecados e
para a vida eterna, para santificação e iluminação, força, cura e saúde da alma
e do corpo, e para consumir e destruir completamente os meus maus pensamentos,
as intenções, os preconceitos e fantasias noturnas inspiradas pelos espíritos
malignos; pois a Ti pertence o reino, o poder, e a glória, e a honra e o louvor,
com o Pai e o Teu Espírito Santo, agora e sempre e pelos séculos dos séculos.
Amém.

\section*{Outra oração de São João Damasceno, 10}

Estou de pé diante das portas do Teu templo pois ainda não me libertei dos meus
pensamentos malignos. Ó Cristo Deus, Tu que justificaste o publicano, que
tiveste misericórdia da mulher de Canaã, que abriste as portas do paraíso ao
ladrão arrependido, abre para mim as profundezas do Teu amor pela humanidade e
recebe-me, a mim que venho e toco em Ti, como recebeste a mulher pecadora e a
que tinha um fluxo de sangue, pois uma foi facilmente curada ao tocar a orla das
Tuas vestes e a outra por tocar Teus pés teve os pecados absolvidos. Indigno que
sou, ouso receber Teu Corpo inteiro. Não me deixes ser consumido pelo fogo, mas
como Tu as recebeste, recebe-me também, e ilumina os meus sentidos espirituais,
queimando meus erros pecaminosos; pela intercessão daquela que sem semente
deu-Te à luz, e dos poderes celestes, pois Tu és bendito pelos séculos dos
séculos. Amém.

\section*{Outra oração de S. J. Crisóstomo}

Creio e confesso, Senhor, que Tu és, em verdade, o Cristo Filho de Deus vivo,
vindo a este mundo para salvar os pecadores, dos quais eu sou o primeiro. Creio
também que estes Dons são o Teu santíssimo e puríssimo Corpo e o Teu Sangue
venerável e precioso.

Eu Te peço, pois: tem piedade de mim e perdoa-me todas faltas voluntárias e
involuntárias, cometidas por palavras e atos, consciente ou inconscientemente e
torna-me digno de participar, sem incorrer em condenação, nos Teus puríssimos
Mistérios, para a remissão dos meus pecados e para a vida eterna. Amém!

\dotextit{Ao aproximar-se da Comunhão, dizer consigo estas palavras de S.
Metafrastes:}

Vê, aproximo-me da Divina Comunhão.
Ó Criador, não me deixes queimar pela comunhão.
Pois Tu és o Fogo, que consome aquele que é indigno.
Antes, purifica-me de toda a impureza.

\dotextit{Depois dizer de novo:}

Recebe-me, Senhor, neste dia, na Tua mística Ceia: eu não desvendarei os
Mistérios aos Teus inimigos: eu não Te darei um beijo como Judas, mas, como o
ladrão arrependido, eu Te confesso: lembra-Te de mim, Senhor, no Teu Reino.

\dotextit{A seguir, estas linhas:}

Teme, ó mortal, ao ver o deificante Sangue;
Pois que Ele é o fogo que consome o indigno.
O Divino Corpo deifica-me e nutre-me.
Ele deifica o espírito e maravilhosamente nutre o pensamento.

\dotextit{A seguir, os Tropários:}

Tu me adoçaste com Teu amor, Ó Cristo, e com Teu zelo, Tu me transformaste.
Consome meus pecados no fogo imaterial e concede-me ser tomado de alegria em Ti.
Que eu possa, jubiloso, glorificar as Tuas duas vindas. Na radiante companhia
dos Teus santos, como posso eu, o indigno, entrar? Pois se ouso entrar na câmara
nupcial, minhas vestes me traem, pois que não são próprias para as bodas, e
então serei lançado para fora por Teus anjos. Ó Senhor, limpa a minha alma
poluída e salva-me, pois Tu amas a humanidade.

\dotextit{A seguir, esta oração:}

Ó Soberano que amas a humanidade, Senhor Jesus Cristo, meu Deus, não deixes que
estes Santos Dons sejam motivo de julgamento, pois sou indigno, mas sirvam para
a purificação e a santificação da alma e do corpo, como uma promessa de vida e
do reino que hão de vir. Pois é bom para mim ser fiel a Deus, e ter no Senhor a
esperança da minha salvação.

\dotextit{E de novo:}

Recebe-me, Senhor, neste dia, na Tua mística Ceia: eu não desvendarei os
Mistérios aos Teus inimigos: eu não Te darei um beijo como Judas, mas, como o
ladrão arrependido, eu Te confesso: lembra-Te de mim, Senhor, no Teu Reino.

\end{document}
