\documentclass{subfiles}

\begin{document}

\chapter{Orações de Agradecimento pela Comunhão}

\dotextit{Quando tiver recebido a Comunhão dos Dons Místicos e Vivificantes, dê
graças imediatamente, agradeça com louvor e diga com ardor, do fundo da alma, as
rações Orações abaixo. As orações podem ser realizadas, por cada um
isoladamente, logo após a comunhão como forma de agradecimento, ou serem
realizadas no final de tudo por clero e todos os fiéis em conjunto.}

\priest{}Glória a Ti, ó Deus! \thrice{}

\section*{Primeira Oração}

\reader{}Agradeço-Te, Senhor meu Deus, por não teres rejeitado a mim pecador,
mas por permitir-me ser participante nos Teus Santos Mistérios. Agradeço-Te
por me ter permitido, embora indigno, receber os Teus Puríssimos Dons
Celestiais. Porém, Senhor Misericordioso, que por nós morreste e ressuscitaste
dentre os mortos, e nos destes estes temíveis e Vivificantes Mistérios para o
bem e a santificação de nossas almas e de nossos corpos, concede que em mim
eles sejam cura para o corpo e para a alma, libertação de todo inimigo,
iluminação dos olhos do meu coração, paz das minhas forças espirituais, uma
fé inquebrantável, um amor sincero, plenitude da sabedoria, cumprimento dos
Teus mandamentos, aumento da Tua divina graça e concede-me encontrar-me
contigo em Teu reino para que, mantido por Eles na Tua Santidade, sempre me
lembre da Tua graça e não viva mais para mim, mas para Ti, nosso Senhor e
Benfeitor. E assim, ao partir dessa vida na esperança da vida eterna, possa
alcançar o descanso eterno, onde é constante a voz dos que se rejubilam e
incessante a felicidade dos que contemplam a beleza indizível de Tua face. Pois
Tu és, o Cristo nosso Deus, o desejo verdadeiro e a felicidade inefável dos que
Te amam, e a Ti louva toda a criação, por toda a eternidade. Amém.

\section*{Segunda Oração, de São Basílio o Grande}

Ó Cristo nosso Deus e nosso Rei, Senhor dos séculos e Criador de todas as
coisas, agradeço-Te por todos os bens que me tens dado e pela comunhão dos Teus
puríssimos e vivificantes Mistérios. Rogo-Te, misericordioso e amigo dos homens,
guarda-me sob Tua proteção, à sombra das Tuas asas e concedame, até o meu último
suspiro, com consciência pura, receber dignamente os Teus Santos Dons para a
remissão dos pecados e alcançar a vida eterna. Pois tu és o Pão da Vida, a Fonte
da Santidade, Doador de Bens. Nós Te glorificamos juntamente com o Pai e o
Espírito Santo, agora e sempre e pelos séculos dos séculos.

\section*{Terceira Oração, de São Simeão Metafrastes.}

Por Tua vontade me deste o Teu Corpo em alimento, ó Fogo que
consome os indignos, não me consumas, o meu Criador. Porém, entra em
meus membros, em todo o meu ser, no coração e na alma. Queima os espinhos
de todos os meus pecados. Purifica a alma, santifica os pensamentos, firma as
ligaduras juntamente com os ossos. Ilumina os meus cinco sentidos, fixa todo
o meu ser no Teu amor. Guarda, protege e livra-me sempre de toda ação ou
palavra destruidora da alma. Purifica-me, lava e adorna: torna-me bondoso,
compreensivo e iluminado. Faça-me morada somente de Teu Espírito, e nunca
do pecado. Que desta Tua casa, pela entrada da Comunhão, fuja, como do
fogo, todo o mal e todo o vício. Apresento diante de ti as orações de todos os
santos, dos seres celestes imateriais, do Teu precursor, dos sábios apóstolos,
juntamente com Tua pura e Imaculada Mãe, cujas preces ó Cristo meu, aceita
com compaixão e faz deste teu servo um filho de luz, Somente Tu trazes, ó
misericordioso, a santidade e a luz às nossas almas e, diariamente nós Te
rendemos glória, que a Ti é devida como Deus e Senhor.

\section*{Quarta Oração}

Que o Teu Santo Corpo seja para a minha vida eterna, ó Senhor Jesus
Cristo, nosso Deus, e o Teu precioso Sangue para a remissão dos pecados. Que
seja este agradecimento para minha alegria, saúde e satisfação e no dia do Teu
terrível segundo advento permita-me ficar à mão direita da Tua glória, pelas
orações de Tua Mãe puríssima e de todos os santos.

\section*{Quinta Oração}

Santíssima Soberana, Mãe de Deus, luz de minha alma obscurecida,
minha esperança, meu abrigo, meu refúgio, minha consolação e alegria:
agradeço-Te por teres permitido que eu, embora indigno, recebesse o
puríssimo Corpo e preciosíssimo Sangue de Teu Filho. Tu, de quem nasceu a
Verdadeira Luz, ilumina os olhos da sabedoria de meu coração. Tendo dado
nascimento à Fonte da Imortalidade, renova minha vida, morta pelo pecado.
Bondosa Mãe de Deus misericordioso, tem misericórdia de mim e dá-me
humildade em meus pensamentos, coração contrito, devoção e liberdade à
minha mente escravizada. Torna-me digno, até o meu último suspiro, de
receber sem condenação a santidade destes puríssimos Sacramentos, para a
cura da alma e do corpo. E dá-me lágrimas de arrependimento e confissão,
para que eu possa louvar-Te e glorificar-Te todos os dias de minha vida, porque
Tu és bendita e gloriosa por todos os séculos. Amém.

Agora, Soberano, deixa o Teu servidor, segundo a Tua palavra partir
em paz, porque os meus olhos viram a Salvação que vem de Ti, que Tu
preparaste para ser apresentada a todos os povos. Luz que brilhará sobre todas
as nações e glória de Teu povo, Israel.

\trisagion{} \thrice{}

\Doxology{}

Santíssima Trindade, tem piedade de nós. Senhor, purifica-nos dos
nossos pecados. Soberano, perdoa-nos as nossas iniquidades. Tu que és Santo,
cura, pelo Teu Nome, as nossas enfermidades e visita-nos.

\mercy{} \thrice{}

\Doxology{}

\ourFather{}

\priest{}Pois Teu é o reino, o poder e a glória, Pai, Filho e Espírito Santo,
agora e sempre e pelos séculos dos séculos.

\dotextit{Seguidamente o Leitor recita o Tropário:}

\reader{}Como uma lâmpada resplandecente, assim brilhou a graça da tua
boca, iluminando o Universo, conservando para o mundo o precioso tesouro
do desprendimento do dinheiro e fazendo-nos ver claramente a excelência da
humildade. Por isso, ó santo Padre João Crisóstomo, cujas palavras edificam
os homens, roga a Cristo, Verbo de Deus, que salve as nossas almas.

\dotextit{Imediatamente recita os Kondákia:}

\doxology{}

Dos Céus recebeste a Graça Divina, ó justo e bem-aventurado João
Crisóstomo. E por aquilo que os teus lábios proferiram, ensinaste a todos a
prostrarem-se diante de Deus, Uno na Trindade. Nós te cantamos, pois, os
devidos louvores, pois não deixarás de ser o Soberano que ilumina os
insondáveis Mistérios Divinos.

\nowandever{}

Ó admirável Protetora dos Cristãos e nossa Medianeira ante o Criador,
não desprezes as súplicas de nenhum de nós pecadores, mas apressa-Te em
auxiliar-nos como Mãe bondosa que és, pois Te invocamos com fé. Roga por
nós junto de Deus, Tu que defendes sempre aqueles que Te veneram.

\mercy{} \repeatn{12}

\Doxology{}

Mais venerável que os Querubins e incomparavelmente mais gloriosa
que os Serafins, deste à luz o verbo de Deus, conservando intacta a tua
virgindade. Nós te glorificamos, ó Mãe de Deus
Sacerdote. Glória a Ti, ó Cristo, nosso Deus e nossa esperança, glória a Ti!

\reader{}\Doxology{}

\mercy{} \thrice{}

Em Nome de Deus, abençoa, Soberano.

\priest{}Pelas orações dos nossos santos Pais, ó Senhor Jesus Cristo, nosso
Deus tem piedade de nós.

\reader{}Amém!

\end{document}
