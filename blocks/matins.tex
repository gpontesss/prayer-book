\documentclass{subfiles}

\begin{document}

\begin{chapterimg}[height=.9\textheight]{icons/tree-of-life-cross}
    \Chapter{Orações Matinais}{}
\end{chapterimg}

\dotextit{%
Tendo despertado do sono, antes de qualquer outra ação, levante-se 
com reverência, considerando estar na presença do Deus que tudo vê, e, tendo 
feito o sinal da Cruz, diga:}
 
Em Nome do Pai, do Filho e do Espírito Santo. Amém. 
 
\dotextit{%
Depois pause por um momento, até que tenha recobrado todos os teus 
sentidos e teus pensamentos abandonem todas as coisas mundanas; e faça 
três pequenas metanóias, dizendo:}
 
Ó Deus, sê misericordioso comigo que sou pecador. 

\section*{Oração Inicial}

Senhor Jesus Cristo, Filho de Deus, pelas orações da Tua puríssima Mãe 
e de todos os Santos, tem piedade de nós. Amém. 
 
Glória a Ti, nosso Deus. Glória a Ti. 
 
\comforter{}
 
\trisagion{} \thrice{}

\Doxology{}

\holytrinity{}
 
\mercy{} \thrice{}

\Doxology{}
 
\ourFather{}

\Section*{Tropário}{para Santíssima Trindade}

Tendo despertado do sono, nós nos prostramos diante de Ti, ó 
Bondoso, e cantamos alto o Hino Angelical a Ti, ó Poderoso: Santo, Santo, Santo 
és Tu, ó nosso Deus; pelas orações da Mãe de Deus, tem piedade de nós. 
 
\doxology{}

Foste Tu que me levantaste do meu leito e sono, ó Senhor: Ilumina a minha mente
e coração, e abre meus lábios para que eu Te louve, Trindade Santa: Santo, Santo,
Santo és Tu, ó nosso Deus; pelas orações da Mãe de Deus, tem piedade de nós. 
 
\nowandever{}

O Juiz Supremo virá subitamente, e as obras de cada um serão mostradas. Eis
porque, no meio da noite, nós Te invocamos com temor: Santo, Santo, Santo és Tu,
ó nosso Deus; pelas orações da Mãe de Deus, tem piedade de nós.

\mercy{} \repeatn{12}


\Section*{Oração de São Basílio, o Grande}{para a Santíssima Trindade}

Quando eu me levanto do sono, eu Te agradeço, Trindade Santa, pois em 
Tua Bondade infinita e paciente não Te encolerizaste comigo, pecador e 
indolente, nem me destruíste pelas minhas iniquidades, mas mostraste o Teu 
usual amor pelo homem; e, quando eu estava prostrado em desespero, Tu me 
levantaste para contemplar o amanhecer e glorificar o Teu poder. Ilumina 
agora o olho de minha mente e abre minha boca para que eu possa meditar 
em Tuas palavras, entender Teus mandamentos, fazer Tua vontade, Te louvar 
com uma confissão sincera, e cantar orações para Teu santíssimo nome: do Pai, 
do Filho, e do Espírito Santo, agora e sempre e pelos séculos dos séculos. 
Amém. 

\letusworship{}

\section*{Salmo 50}

Tem piedade de mim, ó Deus, segundo Tua grande misericórdia; e, segundo a
multidão das Tuas clemências, apaga a minha iniquidade. Lava-me da minha culpa,
e purifica-me do meu pecado. Porque eu conheço a minha maldade, e o meu pecado
está sempre diante de mim. Pequei contra Ti só, e fiz o mal diante dos Teus
olhos, para que sejas encontrado justo nas Tuas palavras, e venças quando fores
julgado. Eis que fui concebido em iniquidades, e minha mãe concebeu-me no
pecado. Eis que Tu amaste a verdade, e me revelaste o segredo e o mistério da
Tua sabedoria. Tu me borrifarás com o hissopo e serei purificado; lavar-me-ás e
me tornarei mais branco que a neve. Tu me farás ouvir uma palavra de gozo e
alegria; regozijar-se-ão meus ossos humilhados. Aparta o Teu rosto dos meus
pecados, e apaga todas as minhas iniquidades. Cria em mim, ó Deus, um coração
puro, e renova nas minhas entranhas um espírito reto. Não me arremesses da Tua
presença, e não tires de mim Teu Espírito Santo. Dá-me a alegria da Tua
salvação, e conforta-me com Teu Espírito magnânimo. Ensinarei aos iníquos Teus
caminhos, e os ímpios se converterão a Ti. Livra-me do sangue, ó Deus, Deus da
minha salvação, e minha língua exaltará a Tua justiça. Senhor, abrirás os meus
lábios, e a minha boca anunciará os seus louvores. Porque, se quisesses um
sacrifício, eu o teria oferecido; mas Tu não Te comprazes com holocaustos. O
sacrifício digno de Deus é um espírito compungido; não desprezarás, ó Deus, um
coração contrito e humilhado. Senhor, sê benigno com Sião por Tua boa vontade,
para que se edifiquem os muros de Jerusalém. Então aceitarás os sacrifícios
legítimos, as oferendas e os holocaustos; então, sobre Teu altar, novilhos serão
colocados.

\section*{O Símbolo da Fé Ortodoxa}

Creio em um só Deus, Pai Onipotente, Criador do Céu e da Terra e de
todas as coisas visíveis e invisíveis.

E em um só Senhor, Jesus Cristo, Filho Unigênito de Deus, nascido do Pai antes
de todos os séculos. Luz de Luz, Deus verdadeiro de Deus verdadeiro, gerado e
não feito, consubstancial ao Pai, por Quem foram feitas todas as coisas. O Qual,
por causa de nós homens e por causa de nossa salvação, desceu dos céus e se
encarnou pelo Espírito Santo e da Virgem Maria e se fez homem. E foi crucificado
por nossa causa, sob o poder de Pôncio Pilatos; padeceu e foi sepultado e
ressuscitou ao terceiro dia, segundo as Escrituras. E subiu aos céus e está
sentado à direita do Pai e novamente virá com glória para julgar os vivos e os
mortos, e cujo Reino não terá fim.

E no Espírito Santo, Senhor Vivificante, que do Pai procede e que é
juntamente com o Pai e o Filho adorado e glorificado, e que falou pelos
profetas.

E na Igreja Una, Santa, Universal e Apostólica. Confesso um só
batismo para a remissão dos pecados. Espero a ressurreição dos mortos e a
vida do século futuro. Amém.

\Section*{Primeira Oração}{de São Macário o Grande}

Senhor, purifica o pecador que eu sou, pois nunca fiz algo de bom à Tua vista;
livra-me do maligno e que a Tua vontade seja feita em mim e que eu possa, sem
condenação, abrir a minha indigna boca e louvar o Teu Santo nome, do Pai, do
Filho e do Espírito Santo, agora e sempre e pelos séculos dos séculos. Amém.

\Section*{Segunda Oração}{do mesmo santo}

Tendo levantado do sono eu ofereço a Ti ó Salvador o hino da meia noite, e
caindo eu clamo a Ti: Concede-me não dormir na morte do pecado, mas tem
compaixão de mim ó Tu que foste voluntariamente crucificado, e apressa-Te em
levantar-me eu que estou reclinado na indolência e salva-me em oração e
intercessão; e após o sono da noite faz brilhar sobre mim um dia sem pecado, ó
Cristo Deus, e salva-me.

\Section*{Terceira Oração}{do mesmo santo}

Tendo levantado do sono, corro para Ti, ó Soberano que amas a humanidade, e pelo
Teu amor esforço-me para executar a Tua obra, e rogoTe: ajuda-me em todo o
tempo, em tudo, e liberta-me de todas as coisas mundanas, ruins e de todo
impulso do demônio, e salva-me. E conduz-me ao Teu reino eterno. Pois Tu És o
meu criador, o doador e provedor de tudo o que é bom, e em Ti está toda a minha
esperança e para Ti eu dou glória, agora e sempre e pelos séculos dos séculos.
Amém.

\Section*{Quarta Oração}{do mesmo santo}

Senhor, Tu que em Tua abundante bondade e em Tua imensa compaixão concedeste a
mim, Teu servo, atravessar a noite passada sem sofrer ataque de nenhum
adversário maligno, ó Soberano e criador de todas as coisas, concede-me que pela
Tua verdadeira luz e com o coração iluminado, eu faça a Tua vontade, agora e
sempre pelos séculos dos séculos. Amém.

\Section*{Quinta Oração}{de São Basílio o Grande}

Senhor Todo-Poderoso, Deus das Hostes e de toda a carne, que habitas nas alturas
e olhas o que está abaixo, que sondas os corações e o íntimo de cada ser, que
conheces, claramente antecipadamente, os segredos dos homens; ó Luz não-criada e
eterna, onde não há variação nem sombra de mudança, recebe nossas súplicas, ó
Rei Imortal, que nós ousamos fazer por conta da imensidão da Tua compaixão, e
que Te oferecemos agora através de nossos lábios impuro; perdoa-nos dos nossos
pecados por atos, palavras e pensamentos, cometidos com nosso conhecimento ou em
nossa ignorância, e purifica-nos de toda mancha da carne e do espírito.

Concede-nos atravessar a noite da vida presente com o coração atento e a mente
sóbria, sempre esperando a vinda do brilhante e marcado dia do Teu Filho Único,
nosso Senhor e Deus Salvador Jesus Cristo, em que o Juiz de todos virá em glória
para compensar cada um segundo suas obras. Que não sejamos encontrados caídos e
sonolentos, mas vigilantes, ativos e prontos para acompanhá-lo com alegria ao
palácio divino de Sua Glória, onde há sempre o som de festa daqueles que mantém
o festival, e inomináveis delicias para aqueles que contemplam a inefável beleza
do Teu semblante. Pois Tu És a verdadeira Luz que iluminas e santificas tudo, e
toda a criação Te louva pelos séculos dos séculos. Amém.

\Section*{Sexta Oração}{também de São Basílio}

Nós Te bendizemos, Deus Altíssimo e Senhor de misericórdia que
sempre fazes conosco grandes e inescrutáveis coisa, gloriosas e magníficas
para as quais não há medida. Tu que concede-nos dormir para o repouso de
nossas enfermidades e relaxamento dos trabalhos de nossa carne cansada; a
Ti damos graças por não nos teres destruído por nossas iniquidades, mas Tu,
como de hábito, mostraste ao homem o Teu amor, e enquanto em desespero
jazíamos em nossos leitos, Tu nos levantaste para que pudéssemos glorificar o
Teu domínio. Assim, imploramos a Tua infinita bondade: ilumina os olhos do
nosso entendimento e levanta as nossas mentes do pesado sono da indolência;
abre a nossa boca e enche-a com Teu louvor para que possamos, prontamente,
cantar-Te e confessar-Te, Tu que És Deus, glorificado em tudo e por tudo, o Pai
incriado, com Teu Filho Único e Teu Santíssimo, Bom e Vivificante Espírito,
agora e sempre e pelos séculos dos séculos.

\Section*{Sétima Oração}{para a Santíssima Theotokos}

Canto a Tua Graça, Soberana Mãe de Deus, e a Ti oro para que clareie
a minha mente. Ensina-me a caminhar com retidão no caminho dos
mandamentos de Cristo. Fortalece-me para que eu fique desperto em louvor
e afasta o sono do desânimo. Noiva de Deus, por Tuas orações liberta-me, pois
estou preso nas correntes do pecado. Ó portadora do Deus doador de vida
vivifica-me, pois estou amortecido pelas paixões. Portadora da Luz sem
declínio, ilumina minha alma cega. Maravilhoso palácio do Soberano, fazes de
mim a morada do Divino Espírito. Portadora d’Aquele que cura, cura-me das
perpétuas paixões da minha alma. Guia-me nos caminhos do arrependimento,
pois sou sacudido pela tempestade da vida. Livra-me do fogo eterno, e dos
vermes malignos e do tártaro. Não me deixes ser exposto à alegria dos
demônios, culpado que sou de tantos pecados. Rejuvenesce-me, ó Imaculada,
pois envelheço por causa dos meus pecados insensatos. Apresenta-me
intocado pelos tormentos, e ora por mim ao Soberano de tudo e de todos.
Concede-me encontrar as alegrias do céu junto a todos os Santos. Santíssima
Virgem, ouve a voz do teu servo inútil. Concede-me torrentes de lágrimas, ó
Toda Pura, que limpem a minha alma de toda impureza. Ofereço-te,
incessantemente, os gemidos de meu coração; luta por mim, Soberana
Senhora. Aceita minhas súplicas e oferece-as ao Deus compassivo. Tu que
estás acima dos anjos, ergue-me acima dessa confusão do mundo. Tu,
portadora da luz do Tabernáculo Celeste, guia a graça do Espírito em mim.
Embora impuras, ergo minhas mãos e meus lábios em louvor a Ti, ó Imaculada.
Livra-me dos males que corrompem a alma, e intercede com fervor junto a
Cristo, a quem devemos honra e louvor, agora e sempre e pelos séculos dos
séculos. Amém.

\Section*{Oitava Oração}{para nosso Senhor Jesus Cristo}

Ó meu Deus imensamente misericordioso, Senhor Jesus Cristo, pelo Teu infinito
amor desceste do Céu e encarnaste para a salvação de todos. Salvador, uma vez
mais, salva-me pela Tua graça, suplico-Te, pois se Tu me salvasses pelas minhas
obras, não seria graça nem dom, mas antes um dever, pois que Tu és grande em
compaixão e inefável em misericórdia; pois todo aquele que crê em mim, conforme
Tu disseste, ó Cristo, viverá e não verá a morte. Se então salvas os que têm fé
em Ti, vê, eu creio, salva-me Tu que és meu Deus e meu Criador.

Deixa que, ao invés de obras, seja-me imputada a fé meu Deus, pois Tu não
encontrarás obras que possam justificar-me. Que a minha fé baste para responder
pelas minhas obras, para absolver-me, fazer-me participe da Tua glória eterna.
Não permitas, ó Verbo, que Satã apodere-se de mim e vangloriese de ter-me tomado
das Tuas mãos e me envolvido. Que eu deseje ou não, salva-me, Cristo meu
Salvador, antecipa-Te prontamente, pois pereço. Tu És o meu Deus desde o ventre
de minha mãe.

Concede-me Senhor, amar-Te agora com tanto fervor, quanto outrora amei o pecado,
e trabalhar para Ti sem preguiça, diligentemente, como outrora para o enganador
satã. E acima de tudo, que eu possa trabalhar para Ti, meu Senhor e meu Deus,
Jesus Cristo, todos os dias da minha vida, agora e sempre e pelos séculos dos
séculos. Amém.

\Section*{Nona Oração}{para o Santo Anjo da Guarda}

Santo anjo, Tu que ficas ao lado de minha alma miserável e de minha vida
arrebatada, não abandones o pecador que sou, nem Te afastes de mim por causa da
minha intemperança. Não dês espaço ao astuto demônio para conduzir-me através da
violência de meu corpo mortal. Fortalece as minhas pobres e fracas mãos, e
guia-me no caminho da salvação. Sim, Santo anjo de Deus, guardião e protetor da
minha alma miserável e de meu corpo, perdoame tudo em que possa ter-Te ofendido
em todos os dias da minha vida. Se pequei durante a noite que passou, protege-me
durante este dia, e guarda-me de toda tentação do inimigo, para que eu não
desagrade a Deus com algum pecado; e roga ao Senhor por mim. Que Ele me
estabeleça no temor por Ele, e que eu, Seu servo, seja digno de Sua bondade.
Amém.

\Section*{Décima Oração}{para a Santíssima Theotokos}

Ó minha Santíssima Senhora Theotokos, por Tuas Santas e poderosas orações,
retira de mim, teu servo vil e miserável, o desânimo, o esquecimento, a
insensatez, a negligência, e todo o pensamento impuro, maligno e blasfemo do meu
coração miserável e da minha mente obscurecida. Apaga o fogo das minhas paixões,
pois sou um pobre miserável. Livra-me das cruéis lembranças e atos, assim como
de seus efeitos malignos. Pois Tu És bendita por todas as gerações, e
glorificado é o teu nome pelos séculos dos séculos. Amém.

\section*{Oração para o Santo Cujo Nome Possuímos}

Ora junto á Deus por mim, ó Santo (nome), bem-aventurado de Deus,
pois com fervor imploro a ti, propiciador e intercessor pela minha alma.

\section*{Oração para a Santíssima Theotokos}

Rejubila, ó Virgem Mãe de Deus, ó Maria cheia de graça, pois o Senhor é contigo,
bendita És Tu entre as mulheres e bendito é o Fruto do Teu Ventre, pois deste a
luz ao Salvador de nossas almas.

\section*{Tropário da Cruz}

Salvar, Senhor, o Teu Povo e abençoa a Tua herança. Concede aos Cristãos
Ortodoxos a vitória sobre os seus inimigos e, pela Tua Cruz, protege as nossas
cidades.

\section*{Oração dos monges de Optina}

Senhor, concede-me a graça de saber aceitar tudo que venha acontecer neste dia
que se inicia. Permita que eu me entregue completamente à Tua Santa vontade em
todo momento deste dia. Ajuda-me e orienta-me em tudo em todos os meus atos e
palavras. Guia meus pensamentos e sentimentos em todos os casos inesperados. Não
permita que eu me esqueça que tudo vem de Ti. Ensina-me a agir corretamente com
cada membro da minha família para que não ofenda e não magoe ninguém. Senhor,
dá-me forças para superar o cansaço deste dia, e suportar tudo o que hoje venha
acontecer. Dirija a minha vontade, e ensina-me a rezar, ter fé, esperança,
paciência, saber perdoar e amar. Amém.

\dotextit{Durante a Santa e Grande Quaresma dizer:}

\section*{Oração de arrependimento Santo Efrem, o Sírio}

Senhor e Mestre da minha vida, afasta de mim o espírito da preguiça, do desânimo,
do domínio e da vã loquacidade.

Concede a teu servo o espírito da castidade, da humildade, da paciência e do amor.

Sim, Senhor e Rei, concede-me que eu veja as minhas faltas e que não julgue o meu
irmão, pois Tu és bendito pelos séculos dos séculos. Amém.

Deus, purifica-me, que sou pecador. \repeatn{12}

Senhor e Mestre da minha vida, afasta de mim o espírito da preguiça, do desânimo,
do domínio e da vã loquacidade.

Concede a teu servo o espírito da castidade, da humildade, da paciência e do amor.

Sim, Senhor e Rei, concede-me que eu veja as minhas faltas e que não julgue o meu
irmão, pois Tu és bendito pelos séculos dos séculos. Amém.

\dotextit{%
Ofereça agora uma pequena oração pela saúde e salvação do seu pai espiritual,
seus pais, familiares, governantes, benfeitores e outros conhecidos, os
enfermos, ou aqueles que estão passando por algum sofrimento.}

\dotextit{E, se possível, leia esta comemoração:}

\section*{Pelos Vivos}

Lembra-Te, Senhor Jesus Cristo, nosso Deus, das Tuas misericórdias e da Tua
compaixão que são pelos séculos dos séculos, pelas quais Tu Te tomaste homem e
quiseste suportar a crucifixão e a morte pela salvação daqueles que corretamente
creem em Ti, Tu que tendo ressuscitado dos mortos ascendeste aos céus onde estás
sentado à direita de Deus Pai, vê as humildes súplicas daqueles que Te clamam de
todo o coração. Inclina o Teu ouvido e ouve as minhas humildes súplicas, teu
servo inútil, como um perfume de espiritual suavidade, que Te ofereço por todo o
Teu povo. Lembra-te primeiro da Tua Igreja Santa, Católica e Apostólica, que Tu
proporcionaste pelo Teu precioso sangue; estabelece, fortalece, expande,
aumenta, pacifica e mantêm-na invencível aos portões do hades. Abranda as
dissensões das igrejas, apaga a fúria das nações, destrói prontamente as
sementes da heresia, aniquila-as pelo poder do Teu Espírito Santo.
\metanoia{}

Salva Senhor, e tem piedade de nosso senhor, Sua Eminência Reverendíssima Dom
\name, Metropolita da América Oriental e de Nova York, primaz da Igreja
Ortodoxa Russa na Diáspora, e de nosso senhor, Sua Excelência Reverendíssima Dom
\name, epíscopo de São Paulo e América do Sul, e dos Excelentíssimos e
reverendíssimos metropolitas, arcebispos e bispos ortodoxos, dos presbíteros,
diáconos e de todos quantos servem na Igreja, os quais estabeleceste para
apascentar a Tua grei e, por suas orações, tem piedade e salva-me a mim, um
pecador. \metanoia{}

Salva Senhor, e tem misericórdia do meu pai espiritual \name, e por
suas santas orações perdoa os meus pecados. \metanoia{}

Salva Senhor, e tem misericórdia de meus pais \names, irmãos e irmãs,
familiares segundo a carne, todos os vizinhos de minha família e amigos,
concedendo-lhes Teus bens terrenos e espirituais. \metanoia{}

Salva Senhor, e tem misericórdia, do idoso e do jovem, do pobre , dos órfãos e
das viúvas, daqueles que sofrem por doença ou tristeza, infortúnio e tribulação,
os que estão em situação difícil e cativeiro, nas prisões e nos calabouços, e
especialmente, daqueles Teus servos, que são perseguidos pelos povos sem Deus,
pelos apóstolas e pelos heréticos, por salvaguardarem a Ti e à Fé ortodoxa.
Lembra-Te deles, visita-os, fortalece-os, conforta-os e pelo Teu poder,
concede-lhes prontamente o alivio, a liberdade e libertação. \metanoia{}

Salva Senhor, e tem piedade daqueles que me odeiam, me enganam e que me tentam,
que eles não pereçam por culpa do pecador que eu sou. \metanoia{}

Ilumina com a luz da consciência, os apóstatas da Fé ortodoxa, e aqueles que
foram cegados pelas heresias perniciosas, e junta-os à Tua Santa, Apostólica e
Católica Igreja. \metanoia{}

\section*{Pelos Nascido para o Céu}

Lembra-Te Senhor, daqueles que partiram desta vida, Reis e Rainhas ortodoxos,
Príncipes e Princesas, Santíssimos Patriarcas, Reverendíssimos Metropolitas,
Arcebispos e Bispos ortodoxos, aqueles das ordens sacerdotais e clericais da
Igreja, e aqueles que Te serviram nas ordens monásticas. Concede-lhes o repouso
junto a Teus Santos, nos Teus tabernáculos eternos. \metanoia{}

Lembra-Te Senhor, das almas dos Teus servos defuntos, meus pais \names, e de
minha parentela segundo a carne. Perdoa-lhes as transgressões voluntárias e
involuntárias. Concede-lhes o reino, uma parte dos Teus bens eternos, e as
delicias da Tua vida eterna e bendita. \metanoia{}. Lembra-Te Senhor,
também dos nossos pais, irmãos e irmãs que aqui repousam, e de todos os cristãos
Ortodoxos que partiram na esperança da ressurreição e da vida eterna.
Estabelece-os junto aos Teus Santos, onde a luz da Tua Face brilhará sobre eles,
e tem piedade de nós, pois Tu És bom e amas a humanidade. \metanoia{}

Concede, Senhor, a remissão dos pecados aos nossos pais, irmãos e irmãs que
partiram antes de nós na fé e na esperança da ressurreição. A eles, memória
eterna. \metanoia{}

\section*{Oração Final}

Verdadeiramente, é digno e justo que Te bendigamos, ó Bem-Aventurada Mãe de
Deus. Tu mais venerável que os Querubins e incomparavelmente mais gloriosa que
os Serafins, deste a luz o Verbo de Deus, conversando intacta a Tua Virgindade.
Nós Te glorificamos, ó Mãe do nosso Deus.

\Doxology{}

\mercy{} \thrice{}

Em nome do Senhor, abençoa Soberano.

Senhor Jesus Cristo, Filho de Deus, pelas orações da Tua puríssima Mãe, dos
nossos Santos e Teóforos Padres, e de todos os Santos, tem piedade de nós. Amém.

\end{document}
