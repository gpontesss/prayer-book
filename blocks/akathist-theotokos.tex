\documentclass{subfiles}

\begin{document}

\begin{chapterimg}[width=\textwidth]{icons/platytera}
    \chapter{Acatiste à Mãe de Deus}
\end{chapterimg}

\section*{Kontákion 1, tom 8}

A ti, Protetora Invencível, nós, Teus servos, dedicamos esta festa de
vitória e ação de graças por termos sido resgatados dos nossos sofrimentos, ó
Mãe de Deus. Tu cujo poder é invencível, livra-nos de todos os perigos que
possam nos ameaçar. Que possamos clamar-te: Rejubila, Esposa Inesposada!

\section*{Ikos 1}

O Anjo foi enviado do Céu para dizer à Mãe de Deus: Rejubila, e cheio
de admiração, ao ver que, a esta palavra imaterial, o Senhor encarnou,
permaneceu de pé junto d'Ela, clamando assim:
Rejubila, Esplendor de alegria!
Rejubila, Destruidora da maldição!
Rejubila, Reabilitação de Adão caído!
Rejubila, Fim das lágrimas de Eva!
Rejubila, Cume inacessível ao pensamento humano!
Rejubila, Abismo impenetrável aos próprios olhos dos anjos!
Rejubila, Trono do Rei celeste!
Rejubila, Portadora daquele que contém tudo!
Rejubila, Estrela anunciadora do Sol!
Rejubila, Seio da encarnação divina!
Rejubila, Renovadora de toda a criatura!
Rejubila, Tu em quem nós adoramos o Criador!
Rejubila, Esposa inesposada!


\section*{Kontákion 2}

A Toda Santa, conhecendo Sua pureza, dizia a Gabriel: Tua palavra tão
gloriosa é difícil ser admitida por Minha alma, pois como falas tu de um
nascimento sem concepção ordinária, clamando: Aleluia!

\section*{Ikos 2}

A Virgem, tentando compreender o que é inacessível à razão, dizia ao
Anjo: "Diz-Me como é que de um seio imaculado pode nascer um Filho?" E, ele
cada vez com mais veneração, chamava-A assim:
Rejubila, Mistério de indizível Sabedoria!
Rejubila, Fé dos que procuram o silêncio!
Rejubila, Princípio dos milagres de Cristo!
Rejubila, Cabeça de Seus mandamentos!
Rejubila, Escada celeste pela qual desce Deus!
Rejubila, Ponte que conduz ao Céu aqueles que estão na terra!
Rejubila, Milagre proclamado pelos anjos!
Rejubila, Ferida chorosa dos demônios!
Rejubila, Geradora da Luz indizível!
Rejubila, Tu que não revelaste Teu segredo a alguma carne!
Rejubila, Cume que ultrapassa a razão dos maiores sábios!
Rejubila, Iluminadora do espírito dos fiéis!
Rejubila, Esposa inesposada!

\section*{Kontákion 3}

O poder do Altíssimo cobriu com Sua sombra a Esposa não desposada
para a tornar fecunda e revelou em Seu seio Sua doce morada, fonte de
salvação para todos aqueles que cantam: Aleluia!

\section*{Ikos 3}

A Virgem levando Deus em Seu seio foi a casa de Isabel cujo filho,
reconhecendo Aquela que saudava sua mãe, rejubilou cantando à Mãe de
Deus:

Rejubila, Ramo da cepa incorruptível!
Rejubila, Colheita do Fruto imortal!
Rejubila, Autora do Benfeitor dos homens!
Rejubila, Geradora do Semeador de nossa vida!
Rejubila, Campo que produz abundância dos benefícios!
Rejubila, Banquete que oferece a plenitude da pureza!
Rejubila, Floração do Paraíso que nos alimenta!
Rejubila, Refúgio das nossas almas!
Rejubila, Louvor agradável de orações!
Rejubila, Purificação do Universo!
Rejubila, Benevolência de Deus para com os mortais!
Rejubila, Audácia dos mortais perante Deus!
Rejubila, Esposa inesposada!

\section*{Kontákion 4}

O casto José, interiormente perturbado por uma tempestade de
dúvidas, sabendo-Te sem esposo, julgou-Te culpada, ó Toda Pura, mas tendo
aprendido que Tu havias concebido pelo Espírito Santo, exclamou: Aleluia!

\section*{Ikos 4}

Os pastores, ouvindo os anjos cantarem a vinda do Senhor encarnado,
acorreram para Ele como para seu Pastor, e vendo-O como um puro Cordeiro
alimentado por Maria, cantaram-Lhe, dizendo:

Rejubila, Mãe do Cordeiro e do Pastor!
Rejubila, Redil das ovelhas espirituais!
Rejubila, Tormento dos inimigos invisíveis!
Rejubila, Acesso às portas do Paraíso!
Rejubila, Tu por quem os Céus rejubilam com a terra!
Rejubila, Tu por quem a Terra rejubila com os Céus!
Rejubila, Boca jamais silenciosa dos Apóstolos!
Rejubila, Firmeza invencível dos Confessores!
Rejubila, Afirmação inabalável da Fé!
Rejubila, Ciência radiosa de graça!
Rejubila, Tu por quem se despoja o Inferno!
Rejubila, Tu por quem nós nos revestimos de glória!
Rejubila, Esposa inesposada!

\section*{Kontákion 5}

Os Magos tendo visto a estrela divinamente dirigida, seguiram a via
luminosa e, tendo-a diante deles como luzeiro, conheceram por ela o Rei
poderoso e, repletos de alegria, cantaram-Lhe: Aleluia!

\section*{Ikos 5}

Os filhos dos Caldeus viram nos braços da Virgem Aquele cuja mão
criou o Homem e reconhecendo n'Ele o Senhor, ainda que dissimulado sob a
aparência de servo, apressaram-se a servi-Lhe com a oferta de presentes,
exclamando àquela que é bem-Aventurada:

Rejubila, Mãe da Estrela sem crepúsculo!
Rejubila, Aurora do dia misterioso!
Rejubila, Extinção da fornalha de sedução!
Rejubila, Tu que iluminas o mistério da Trindade!
Rejubila, Tu que derrubas o domínio do tirano cruel!
Rejubila, Receptáculo de Cristo Senhor, Amigo do homem!
Rejubila, Tu que nos livras da servidão dos bárbaros!
Rejubila, Tu que nos libertas das obras das trevas!
Rejubila, Tu que apagas a adoração do fogo!
Rejubila, Tu que apaziguas o fogo das paixões!
Rejubila, Guia de castidade para os fiéis!
Rejubila, Alegria de todas as gerações!
Rejubila, Esposa inesposada!

\section*{Kontákion 6}

Os Magos portadores da mensagem divina voltaram para Babilônia
tendo cumprido a profecia e proclamando a Cristo perante todos,
abandonaram a falsidade de Herodes que não quisera aprender com eles a
cantar: Aleluia!

\section*{Ikos 6}

Tendo brilhado no Egito, Luz de Verdade, Tu destruíste as trevas da
mentira, pois seus ídolos, ó Salvador, não tendo podido resistir à Tua força,
caíram. Livres deles, nós cantamos à Mãe de Deus:

Rejubila, Reparação da humanidade!
Rejubila, Ruína total dos demônios!
Rejubila, Tu que destróis o poder sedutor!
Rejubila, Tu que confundiste a mentira dos ídolos!
Rejubila, Mãe que devoras o Faraó do espírito!
Rejubila, Pedra que mataste a sede aos sequiosos da vida!
Rejubila, Coluna de fogo que guia nas trevas!
Rejubila, Abrigo do mundo, maior do que o firmamento!
Rejubila, Alimento e reserva do Maná celeste!
Rejubila, Oferenda de alegria santa!
Rejubila, Terra da promessa!
Rejubila, Tu de quem brotam o leite e o mel!
Rejubila, Esposa inesposada!

\section*{Kontákion 7}

Quando Simeão desejava deixar este mundo sedutor, Tu apareceste a
seus olhos sob o aspecto de uma Criança e ele reconheceu em Ti o Deus de
perfeição. Venerando a Tua sabedoria indizível, ele exclamou: Aleluia!

\section*{Ikos 7}

O Criador mostrou-nos uma obra nova da criação aparecendo-nos a
nós criados por Ele, germinando de um seio inviolado e conservando-o
imaculado, a fim de que, contemplando este milagre, nós cantemos a Virgem,
dizendo:

Rejubila, Flor de incorrupção!
Rejubila, Coroa de castidade!
Rejubila, Esplendor da Ressurreição!
Rejubila, Imagem da vida angélica!
Rejubila, Aurora de frutos de luz, alimento dos fiéis!
Rejubila, Árvore de folhagem benfazeja onde muitos se abrigam!
Rejubila, Tu cujas entranhas levaram o Libertador dos cativos!
Rejubila, Geradora do Guia dos transviados!
Rejubila, Tu que alcanças misericórdia do Juiz de equidade!
Rejubila, Perdão dos pecados!
Rejubila, Veste de coragem para aqueles que estavam nus!
Rejubila, Amor vencedor de todos os desejos!
Rejubila, Esposa inesposada!

\section*{Kontákion 8}

Contemplando o nascimento miraculoso, afastemos nossos
pensamentos mundanos, elevando-nos para os Céus, pois é para estes fins que
Deus supremo apareceu sobre a terra como um humilde humano para atrair
para as alturas aqueles que Lhe cantam: Aleluia!

\section*{Ikos 8}

O Verbo indescritível esteve nas regiões inferiores sem deixar os Céus,
pois Sua descida foi divina, Sua passagem efetuou- se sem ruptura pela Virgem
divinamente eleita que Lhe deu nascimento e nos ouve exclamar:

Rejubila, Tabernáculo de Deus imenso!
Rejubila, Porta do mistério sagrado!
Rejubila, Perturbação dos infiéis!
Rejubila, Glória conhecida dos fiéis!
Rejubila, Carro sagrado d'Aquele que está sentado acima dos Querubins!
Rejubila, Casa gloriosa dAquele que está sentado acima dos Serafins!
Rejubila, Tu que unes o que estava disperso!
Rejubila, Tu que unes a virgindade e a maternidade!
Rejubila, Tu que desligas os laços da falta!
Rejubila, Tu que abres o Paraíso!
Rejubila, Chave do reino de Cristo!
Rejubila, Esperança dos bens eternos!
Rejubila, Esposa inesposada!

\section*{Kontákion 9}

Todos os seres angélicos admiravam o grande mistério da Encarnação,
vendo o Deus inacessível feito homem acessível a todos e habitando entre nós,
e ouvindo-nos cantar a todos: Aleluia!

Ikos 9

Os mais ilustres oradores são mudos como os peixes para falarem de
Ti, ó Mãe de Deus, pois eles não podem explicar como, conservando a Tua
virgindade, Tu pudeste dar à luz. E nós, admirando com surpresa este mistério,
Te cantamos com fé:

Rejubila, Tabernáculo da Sabedoria de Deus!
Rejubila, Tesouro da Sua Providência!
Rejubila, Tu que tornas os sábios insensatos!
Rejubila, Tu que convences de contra-senso a astúcia das palavras!
Rejubila, pois os que procuram o mal são confundidos!
Rejubila, pois os mitólogos decaíram!
Rejubila, Tu que rompeste as redes atenienses!
Rejubila, Tu que enches as redes dos pescadores!
Rejubila, Tu que nos afastas dos abismos da ignorância!
Rejubila, Tu que iluminas as inteligências!
Rejubila, Barca daqueles que se querem salvar!
Rejubila, Porto das navegações da vida!
Rejubila, Esposa inesposada!

\section*{Kontákion 10}

O Benfeitor que ornamenta tudo, querendo salvar o mundo, vem a ele
segundo a Sua própria promessa. Deus, nosso Pastor, vem a nós como um
homem, chamando-nos a Ele por esta semelhança e Ele nos ouve cantar-Lhe
como a nosso Deus: Aleluia!

\section*{Ikos 10}

Ó Virgem, Mãe de Deus, Tu és a Muralha de apoio das virgens e de
todos aqueles que recorrem a Ti, pois o Criador dos Céus e da Terra assim o
determinou, ó Toda Pura, entrando em Teu seio e ensinando a todos Te
invocar:

Rejubila, Coluna da virgindade!
Rejubila, Porta da Salvação!
Rejubila, Mestra de edificação espiritual!
Rejubila, Doadora dos bens divinos!
Rejubila, Tu que renovaste aqueles que foram concebidos no pecado!
Rejubila, pois Tu instruíste aqueles cujo espírito se tinha transviado!
Rejubila, Tu que afastas o corruptor dos pensamentos!
Rejubila, Tu que geraste o Semeador de pureza!
Rejubila, Palácio de esponsais imaculados!
Rejubila, União dos fiéis ao Senhor!
Rejubila, requintado Alimento das virgens!
Rejubila, Tu que ornamentas com a veste nupcial as almas santas!
Rejubila, Esposa inesposada!

\section*{Kontákion 11}

É em vão que os nossos cânticos tentam se estender à multidão de
Teus numerosos benefícios, ó Rei Santo; mesmo que nós Te oferecêssemos tão
numerosos como os grãos de areia, jamais alcançaríamos de uma maneira
digna o que Tu nos deste a nós que Te cantamos: Aleluia!

\section*{Ikos 11}

É a chama acesa iluminando aqueles que estão nas trevas que nós
vemos na Virgem Santa. Ela acende a chama imaterial, Ela ensina o
conhecimento do divino, Ela ilumina o espírito como uma aurora e é Ela que
nós veneramos neste apelo:

Rejubila, Raio de sol espiritual!
Rejubila, Astro de luz sem ocaso!
Rejubila, Relâmpago que iluminas as almas!
Rejubila, Raio que terrifica os inimigos!
Rejubila, Tu que fazes brilhar as luzes radiosas!
Rejubila, Tu que fazes correr rios abundantes!
Rejubila, Imagem viva da água do batismo!
Rejubila, Tu que lavas as manchas do pecado!
Rejubila, Purificação da consciência!
Rejubila, Taça transbordante de alegria!
Rejubila, Odor dos perfumes de Cristo!
Rejubila, Vida de alegria misteriosa!
Rejubila, Esposa inesposada!

\section*{Kontákion 12}

Aquele que apaga os pecados dos homens, tendo querido cobrir com a Sua graça
todas as dívidas antigas, vem Ele mesmo àqueles que se tinham desviado de Sua
graça e, rasgando o rolo de nossos pecados, Ele ouve elevarse de todos nós para
Ele este canto: Aleluia!

\section*{Ikos 12}

Ó Mãe de Deus, nós cantamos Tua maternidade, nós Te glorificamos
como um templo vivo. Com efeito, permanece em Teu seio Aquele que em Sua
mão contém tudo. Santifica-nos, ilumina-nos, ensina-nos a clamar-Te:

Rejubila, Morada do Verbo de Deus!
Rejubila, Santa mais santa que os Santos!
Rejubila, Arca dourada pelo Espírito!
Rejubila, Tesouro inesgotável da vida!
Rejubila, Coroa gloriosa dos monarcas piedosos!
Rejubila, Louvor glorioso dos sacerdotes devotos!
Rejubila, Coluna inabalável da Igreja!
Rejubila, Muro indescritível do Império!
Rejubila, Tu que concedes as vitórias!
Rejubila, Tu que dispersas os inimigos!
Rejubila, Cura do meu corpo!
Rejubila, Salvação da minha alma!
Rejubila, Esposa inesposada!

\section*{Kontákion 13}

Ó Mãe tão cantada, que deste à luz o Verbo mais Santo que os Santos,
recebe agora a nossa oferenda, livra de todo o mal e resgata dos tormentos
futuros todos aqueles que Te clamam: Aleluia, Aleluia, Aleluia! Este Kontákion
é repetido três vezes e logo depois:

\section*{Ikos 1}

O Anjo foi enviado do Céu para dizer à Mãe de Deus: Rejubila, e cheio
de admiração, ao ver que, a esta palavra imaterial, o Senhor encarnou,
permaneceu de pé junto d'Ela, clamando assim:

Rejubila, Esplendor de alegria!
Rejubila, Destruidora da maldição!
Rejubila, Reabilitação de Adão caído!
Rejubila, Fim das lágrimas de Eva!
Rejubila, Cume inacessível ao pensamento humano!
Rejubila, Abismo impenetrável aos próprios olhos dos anjos!
Rejubila, Trono do Rei celeste!
Rejubila, Portadora d'Aquele que contém tudo!
Rejubila, Estrela anunciadora do Sol!
Rejubila, Seio da encarnação divina!
Rejubila, Renovadora de toda a criatura!
Rejubila, Tu em quem nós adoramos o Criador!
Rejubila, Esposa inesposada!

\section*{Kontákion 1}

A ti, Protetora Invencível, nós, Teus servos, dedicamos esta festa de
vitória e ação de graças por termos sido resgatados dos nossos sofrimentos, ó
Mãe de Deus. Tu cujo poder é invencível, livra-nos de todos os perigos que
possam nos ameaçar. Que possamos clamar-te: Rejubila, Esposa Inesposada!

\section*{Oração à Santíssima Mãe de Deus}

Ó Santa e Soberana Mãe de Deus! Tu es mais venerável do que todos os anjos e
arcanjos; Mais venerável que toda a criação, ajuda dos oprimidos, esperança dos
desesperados, intercessora dos pobres, consolo dos que sofrem, alimento dos
famintos, veste daqueles que estão nus, cura dos doentes, salvação dos
pecadores, ajuda e proteção de todos os Cristãos. Ó Misericordiosa Soberana
Virgem Mãe de Deus! Através da Tua misericórdia salva e tem piedade dos santos
patriarcas Ortodoxos, dos veneráveis metropolitas, arcebispos e bispos e de
todos os sacerdotes e aqueles das ordens monásticas, dos líderes militares e
civis, das forças armadas e seus simpatizantes e todos os Cristãos Ortodoxos:
defenda-os pelo Teu precioso omophorion e interceda junto a Cristo, nosso Deus,
que encarnou de Ti a fim de com o Seu poder do alto, Ele nos proteja de nossos
inimigos visíveis e invisíveis. Ó Misericordiosa e Soberana Virgem Mãe de Deus!
Levanta-nos das profundezas do pecado e livra-nos da fome, destruição, dos
terremotos e enchentes, do fogo e da espada, da invasão de seres invisíveis e da
guerra civil, de uma morte repentina e dos ventos malignos, de qualquer praga
mortal e de todo o mal. Conceda, Senhora, paz e saúde aos Teus servos, todos os
Cristãos Ortodoxos e ilumine suas mentes e os olhos de seus corações com a luz
da salvação; Torne a nós, teus servos pecadores, dignos do Reino de Teu Filho,
Cristo nosso Deus, pois glorificado e louvado é o Seu domínio, juntamente com o
Seu Pai sem princípio e sem fim e Seu Santíssimo e vivificante Espírito, agora e
sempre e pelos séculos dos séculos. Amém.

\end{document}
